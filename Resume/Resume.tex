%%%%%%%%%%%%%%%%%%%%%%%%%%%%%%%%%%%%%%%%%%%%%%%%%%%%%%%%%%%%%%%%%%%%%%%%%%%%%%%%
%
% Antonio Kim
% Resume LaTeX source file
%
%%%%%%%%%%%%%%%%%%%%%%%%%%%%%%%%%%%%%%%%%%%%%%%%%%%%%%%%%%%%%%%%%%%%%%%%%%%%%%%%


\documentclass{Resume}

\setname{               Jae Hoon (Antonio) Kim                      }
\setaddress{            Toronto/Canada                              }
\setmobile{             (647) 964-5748                              }
\setmail{               hire@antoniojkim.com                        }
% \setposition{           Machine Learning Frameworks Engineer        }
\setlinkedinname{       antoniojkim                                 }
\setlinkedinaccount{    https://www.linkedin.com/in/antoniojkim/    }
\setgithubname{         antoniojkim                                 }
\setgithubaccount{      https://github.com/antoniojkim              }
% \setthemecolor{red} %you can play with color of the template (red is also nice..)

\begin{document}

\info

\horizontalline


% %Set variables
% %You can add sections, texts, explanations just by copying the style below. Replace the dummy texts "\lipsum[1][x-x]\par" with actual texts.
% %Create header
% \headerview
% \vspace{1ex}
% %Sections
% %
% %Education
% \section{Education} 
%     \datedexperience{Boğaziçi University}{2013-2018} 
%     \explanation{B.S in Computer Engineering} 
%      \explanationdetail{\coloredbullet\ % 
%      \lipsum[1][3-4]\par %replace this part with actual text
%      }
%     \datedexperience{Technical University of Munich}{2019-2022} 
%     \explanation{M.S in Informatics} 
%     \explanationdetail{\coloredbullet\ %
%      \lipsum[1][3-4]\par %replace this part with actual text
%      }
% %
% % Experience
% \section{Experience}
%     %
%     \datedexperience{Intertech Inc.}{2014-Summer / Turkey} 
%     \explanation{Intern as android developer} 
%     %
%     \datedexperience{Velocity Inc.}{2015-Summer / Turkey} 
%     \explanation{Intern as Developer/Tester} 
%     %
%     \datedexperience{Akbank}{2018-2019 / Turkey} 
%     \explanation{Ios Developer} 
%     \explanationdetail{\coloredbullet\ %
%      \lipsum[1][1-2]\par %replace this part with actual text
%      }
%     %
%     \datedexperience{Mobile-Software AG}{2019-2020 / Germany} 
%     \explanation{Ios Developer} 
%     \explanationdetail{\coloredbullet\ %
%      \lipsum[1][4-5]\par %replace this part with actual text
%      }
%     %
%     \datedexperience{BMW Autonomous Driving Campus}{2020-Now / Germany} 
%     \explanation{Working Student} 
%     \explanationdetail{\coloredbullet\ %
%      \lipsum[1][3-4]\par %replace this part with actual text
%      }
% %
% % Skills
% \section{Skills}
%     %
%     \newcommand{\skillone}{\createskill{Programming Languages}{\textbf{\emph{Experienced:}} \ \  Python \cpshalf Swift \ \ \textbf{\emph{Familiar:}} \ \  Javascript \cpshalf Objective-C \cpshalf Bash \cpshalf Java \cpshalf Scheme}}
%     %
%     \newcommand{\skilltwo}{\createskill{Software Development}{Programming Paradigms \cpshalf GIT \cpshalf CLI \cpshalf Agile Methodology \cpshalf DevOps Lifecycles}}
%     %
%     \newcommand{\skillthree}{\createskill{Frameworks \ \& \ Libraries}{Jupyter \cpshalf Matplotplib \cpshalf Numpy \cpshalf Pandas \cpshalf Scikit-learn \cpshalf Gym \cpshalf PyTorch \cpshalf Tensorflow}}
%     %
%     \newcommand{\skillfour}{\createskill{iOS Programming}{RxSwift \cpshalf PromiseKit \cpshalf CocoaPods \cpshalf Autolayout/DSLs}}
%     %
%     \newcommand{\skillfive}{\createskill{Languages}{\textbf{\emph{Native:}} \ \  Turkish \ \ \textbf{\emph{Fluent:}} \ \ English \ \ \textbf{\emph{Beginner:}} \ \  German }}
%     %
%     \createskills{\skillone, \skilltwo, \skillthree, \skillfour, \skillfive}
% %
% % Experience
% \section{Extra}
%     \newcommand{\extraone}{%
%     \lipsum[1][7-8]\par %replace this part with actual text
%     }
%     %
%     \newcommand{\extratwo}{%
%     \lipsum[1][9-10]\par %replace this part with actual text
%     }
%     %
%     \newcommand{\extrathree}{%
%     \lipsum[1][11-12]%replace this part with actual text
%     }
%     %
%     \newcommand{\listofextras}{\extraone, \extratwo, \extrathree}
%     %
%     \createbullets{\listofextras}
% %
% %Footnote
% \createfootnote
\end{document}
